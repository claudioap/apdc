\documentclass[runningheads]{llncs}
\usepackage[utf8]{inputenc}
\usepackage[T1]{fontenc}
\usepackage{verbatim}
\usepackage{graphicx}
\usepackage{hyperref}

\begin{document}
\title{Programming Abstractions for Wireless Distributed Protocols}
\author{Cláudio Afonso de Sousa Pereira}
\institute{NOVA University of Lisbon,\\
Faculdade de Ciências e Tecnolgia, Almada, Portugal\\
\email{cad.pereira@campus.fct.unl.pt}}

\maketitle

\begin{center} Supervised by João Leitão, Carla Ferreira and João Seco \end{center}
\begin{abstract}
We are constantly developing new technologies, with ever growing complexity.
To keep our technologies manageable we try to abstract their workings. The more manageable development is, the further we can develop.
\par Wireless \textit{ad hoc} networks are a relatively new technology, and provide numerous advantages over traditional networks (such as better support for reliable edge computing), but their implementations often depend on complex software solutions. One of those solutions is \textit{Yggdrasil}, a software framework to develop protocols that operate in wireless \textit{ad hoc} networks.
\par Yggdrasil was designed to run on commodity hardware, that is, with efficiency in mind. But efficiency often comes with limitations, and for Yggdrasil that meant low level code which is tedious and error prone.
\par This work will study the design and usage of domain-specific languages tailored to specify wireless ad hoc distributed protocols as an attempt to simplify frameworks such as \textit{Yggdrasil}.
An yet to exist domain specific language named YGGL will be designed specifically to work with the Yggdrasil interfaces.
Ideally it will be flexible enough to allow the programmer to develop with minimal performance penalization over writing native \textit{Yggdrasil} code (in the C programming language), while being considerably less verbose and prone to semantic errors.
\keywords{Distributed protocols  \and Wireless \and Ad hoc \and Abstraction \and Domain specific language \and Transpilation \and YGGL}
\end{abstract}

\section{Introduction}
\subsection*{Problem and objectives}
For the last few decades there has been an explosion in what is marketed as ``\textit{cloud computing}''\cite{Gartner:2017}, the Internet of centralized services in which devices serve as clients that interact with immensely huge data centers.
That approach has its drawbacks, ranging from the technical issues (such as latency), to the social ones (such as privacy), and during this whole transition there have been communities and a few corporations pushing the opposing way, towards the ``\textit{edge}''\cite{Satyanarayanan:2017}. Recently that push has been gathering considerable attention.
\par Edge computing is a system design approach that leaves the computational nodes as near as possible to the nodes that either provide the input data or consume the result.
It provides several benefits over centralized computing, such as low latency, data ownership and better control over hardware.
\par But the edge suffers from the sorites paradox\cite{Leitao:2018}. Where does it end? Which set of machines form the boundary?
The edge has no strict definition. The extent to which a machine is still considered in the edge is up to the system designer. For the purpose of this work we will consider an arbitrarily large wireless \textit{ad hoc} network as our edge scenario.
%\par A network's reliability depends the network topology. Networks based on infrastructure often have a single point of failure, which can be solved with either redundancy (that tends to be expensive). Alternatively, decentralized infrastructure-less networks(such as \textit{ad hoc} networks) do not suffer from that problem as they natively have redundancy.
\par \textit{Ad hoc} networks are modeled as graphs, where links denote two nodes that can communicate directly with each other, and in some scenarios the graph changes dynamically with time\cite{Akos:2018}. They present advantages, such as resilience, and in some scenarios  much cheaper to build. However they are harder to develop, and if set improperly, can have latency issues, and might have issues scaling with the addition of new participants. 
\par In spite of these difficulties, there are several protocols and abstractions such as \textit{Yggdrasil}\cite{Akos:2018}. \textit{Yggdrasil} is a framework that supports the development of wireless \textit{ad hoc} networks.
\textit{Yggdrasil} is a powerful framework, that can run on commodity hardware. Unfortunately, this power and efficiency comes at the cost of simplicity.
\textit{Yggdrasil} is written in the C programming language, a low level language among the general purpose programming languages, and that translates to highly verbose code and a significant burden on the programmer.
\par This work revolves around the development of low cost abstractions on top of \textit{Yggdrasil} as means to study the programming of abstractions for the development of wireless distributed protocols, such as the ones that operate over \textit{ad hoc} networks.
These low cost abstractions will come in the form of domain specific language, \textit{YGGL}.
\subsection*{Document structure}
The remainder of this document is arranged as follows:\\
Section \ref{related} lists existing works that are related to the problem or might help during development; Section 3 proposes an approach that should lead to the desired conclusions; Section 4 presents a work plan, distributing tasks during the available time.

\section{Related work}
\label{related}
Distributed protocols are protocols that work across several processes. They are not simple to program due to the fact that they often operate on irregular environments and asynchronously.
\subsection*{Frameworks for distributed protocols}
There are some softwares to ease the development of such protocols with their abstractions:\\
\textit{Isis}\cite{ISIS:1990}, \textit{Horus}\cite{Horus:1994} and \textit{Appia}\cite{Appia:2001} are frameworks that serve a purpose similar to Yggdrasil, but focusing on group communication without proper support for wireless networking primitives.\\
\textit{TinyOS}\cite{TinyOS:2005} is a light operating system with tooling to program sensor network applications, and  Impala\cite{Impala:2003} is a middleware with a similar purpose.\\
The interfaces present in these softwares might be applicable as a part of the YGGL language.
\subsection*{Domain specific languages}
Domain specific languages(DSL's) are the opposite of general purpose languages (such as C). DSL's are only suitable for a single task. This task specialization allows for better abstractions and optimization without increasing the verbosity.
\par There are several domain specific languages. One that might be a good inspiration source for this work is the ``Packet Language for Active Networks''(\textit{PLAN}). \par \textit{Wireshark} is a network protocol analysis program, and has its own data filter language, which might be useful to filter protocol messages in \textit{YGGL}.
\subsection*{Interpretation process}
There are a few ways to process programming languages, with the two main fields being the interpretation and the compilation of languages.
\par Interpreters are programs that run source code of a given programming language. The program is run either during the interpretation (as lines are parsed), or right after the full program is interpreted, to allow optimization (this process is known as a just-in-time compilation). Unlike interpreters, compilers interpret the source code, but do not run it, outputting a binary file instead. This binary file usually is code that the CPU itself can execute. Since the compilation process does not affect the program runtime, most compilers optimize the resulting program much more than the interpreters do.
\par Interpreters and compilers share many of their techniques. One that is performed often is an intermediate ``transpilation'', where the input source code is outputted as source code in another programming language (like CPU \textit{assembly} or an ``intermediate representation language'').
\par The transpilation process will be useful to convert \textit{YGGL} to \textit{Yggdrasil} native code. There are many techniques which help the process, such as the usage of context free or parsing expression grammars for lexing. There are tools like Flex that help composing textual tokens, and others like Bison that help with the parsing. Even this mature field is still subject to changes, and new tools like Pest appear to be viable alternatives \cite{AppelModernCompiler:1997,FlexBisonLevine:2009,Pest:2019}.

\section{Proposed approach}
\label{proposed}
\subsection{A language for the Yggdrasil framework}
The abstraction issue with \textit{Yggdrasil} comes from the fact that is written in the C programming language. While C is a good programming language for the problem that \textit{Yggdrasil} attempts to solve, it cripples efforts to simplify the framework's usage.
The proposed approach starts with the definition of a domain specific language (\textit{DSL}) tailored for the \textit{Yggdrasil} framework.
DSL's offer a simple form to write optimized code. A good \textit{DSL}, with enough expressiveness, should not significantly degrade the performance of \textit{Yggdrasil}.
\par It is likely that \textit{YGGL} will change during the course of the work, with the specification getting gradually refined.
\subsection{Transpilation}
In order to run an \textit{YGGL} program, the chosen approach is to use a transpilation to native \textit{Yggdrasil} code, in the C programming language.
A perfect transpilation from a zero-cost abstraction language does not lose any performance over a native implementation.
\subsection{Validation}
With the transpiler development being subjected to errors, and the language definition prone to be misinterpreted or unclear, validation helps ensuring conformity with the YGGL specification.
\par A few short programs will be written during the development of the transpiler, as means to test the implemented features, but those examples are short of being enough to declare the transpiler as conformant with the language specification. More extensive validations, such as formal specification languages are one of the possible tools to aid with the validation effort.

\section{Work plan}
This work started in the middle of March 2019, and has a deadline at the June 8\textsuperscript{th} of the same year, totaling about 12 full weeks.
\par These are main tasks at hand:
\begin{enumerate}
	\item Defining a domain specific language for the given use case.
	\item Choosing from the available tooling.
	\item Writing a transpiler.
	\item Writing some example programs in the new domain specific language.
	\item Validating the transpiler using formal testing tools.
	\item Writing a final report.
\end{enumerate}
\subsection{Scheduling}
Given the task numbers above, the proposed scheduling is as follows:
\begin{table}
\centering
\begin{tabular}{c|c|c|c|c|c|c|c|c|c|c|c|c}
\hline
\textbf{Week} & 1 & 2   & 3   & 4 & 5 & 6   & 7   & 8   & 9     & 10 & 11 & 12 \\ \hline
\textbf{Task} & - & 1,2 & 2,3 & 3 & 3 & 3,4 & 3,4 & 3,4 & 3,4,5 & 5  & 5  & 6  \\ \hline
\end{tabular}
\end{table}

\bibliographystyle{plain}
\bibliography{bibliography}
\end{document}
